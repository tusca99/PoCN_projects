\chapter{Task 15: Self-Organized Criticality on Networks}


\section{Framework (BTW sandpile on a graph)}
We simulate the Bak--Tang--Wiesenfeld (BTW) sandpile on a network with $N$ nodes.
Each node $i$ carries an integer load $z_i\in\{0,1,2,\dots\}$ and a threshold $z_c(i)$.
At each driving step one grain is added to a uniformly random node; then the system relaxes via topplings.
On a generic graph we take $z_c(i)=k_i$ (degree threshold), so one toppling at $i$ sends one grain to each neighbor.
If $z_i\ge z_c(i)$, a toppling performs
\begin{equation}
z_i \leftarrow z_i-k_i,\qquad z_j \leftarrow z_j+1\ \ \forall j\in\partial i.
\end{equation}
To reach a stationary regime on finite networks we include dissipation with small probability $f$ (per-toppling) so that avalanches remain finite.
An \emph{avalanche} is the full relaxation triggered by one grain addition.
We measure standard observables~\cite{bonabeau1995sandpile,goh2003sandpile}: size $S$ (total topplings), duration $T$ (number of update waves), and area $A$ (distinct toppled nodes).

\section{Results}
\paragraph{From SOC on networks to SOC with interdependence.}
Power-law avalanche statistics on networks are a standard signature of self-organized criticality in sandpile models~\cite{bonabeau1995sandpile,goh2003sandpile}. Here we focus on a complementary question: how \emph{interdependence} between two networked systems reshapes the probability of extreme events, following Brummitt et al.~\cite{brummitt2012suppressing}.

\paragraph{Coupled sandpiles and large cascades.}
We consider two modules $A$ and $B$ and add sparse interconnections controlled by a coupling parameter $p$.
Here a \emph{module} simply denotes one of the two subnetworks ($A$ or $B$) in the coupled system~\cite{brummitt2012suppressing}.
In this section, $N$ denotes the number of nodes \emph{per module} (so the coupled system has $2N$ nodes).
Because the BTW threshold is degree-based ($z_c(i)=k_i$), adding bridges modifies both (i) pathways for load to propagate between modules and (ii) node capacities/total load that the system can hold~\cite{brummitt2012suppressing}.
\enlargethispage*{1.8\baselineskip}
Following~\cite{brummitt2012suppressing}, we classify events by whether a \emph{large} cascade occurs in module $A$, using a fixed cutoff on the avalanche size in that module: $S_A>C$ with $C=\frac{N}{2}=1000$.
We also track \emph{global} cascades using $S>C_g$ with $C_g=N=2000$, following~\cite{brummitt2012suppressing} like before.
Dissipation is implemented in per-toppling mode with probability $f=0.01$.
The ``regular'' modules are random regular graphs $R(z_a)$ and $R(z_b)$ (here $z_a=z_b=3$), whereas the scale-free modules are generated by a static model with tunable exponent $\gamma$ (including $\gamma=\infty$, approaching a limit to where the degree distribution becomes uniform).
In the scale-free panels, the shaded band shows a 95\% Wilson score confidence interval (binomial) for the estimated probability from aggregated event counts.
The three curves in Fig.~\ref{fig:brummitt_local_compare} separate cascades that stay in $A$ (local) from those triggered by activity arriving from the other module (inflicted), and their sum (any large in $A$). For small $p$, interconnections are beneficial because they suppress the largest within-module cascades; for large $p$, they become detrimental because they enable inflicted events and also increase capacities/total load. The trade-off can produce an intermediate optimum in $p$~\cite{brummitt2012suppressing}.
\enlargethispage*{1.8\baselineskip}
\begin{figure}
    \centering
    \includegraphics[width=0.49\linewidth]{figures/task15/pr_large_in_A_vs_p_pretty_R3.pdf}\hfill
    \includegraphics[width=0.49\linewidth]{figures/task15/pr_large_in_A_vs_p_pretty_SF.pdf}
    \caption{Coupled networks: probability of a large cascade in module $A$ versus coupling $p$ (regular vs scale-free modules). Scale-free modules show a stronger dependence on the generating exponent $\gamma$, while regular modules follow a smoother trend with $p$.}
    \label{fig:brummitt_local_compare}
\end{figure}


\begin{figure}
    \centering
    \includegraphics[width=0.49\linewidth]{figures/task15/pr_large_global_vs_p_pretty_R3.pdf}\hfill
    \includegraphics[width=0.49\linewidth]{figures/task15/pr_large_global_vs_p_pretty_SF.pdf}
    \caption{Robustness across event definitions: probability of a \emph{global} large cascade ($S>C_g$, with $C_g=2000$) versus coupling $p$. The qualitative non-monotonic dependence on $p$ persists, supporting the interpretation of an ``optimal coupling'' rather than a cutoff artifact~\cite{brummitt2012suppressing}.}
    \label{fig:brummitt_global_compare}
\end{figure}

Global large cascades (defined by $S>C_g$) are rarer but more systemic events than ``large-in-$A$'' cascades, as seen in Fig.~\ref{fig:brummitt_global_compare}.
The non-monotonic dependence on coupling $p$ persists: small $p$ can suppress the largest within-module events, whereas large $p$ enables inter-module propagation (inflicted cascades) and increases total capacity/load, which can fuel larger system-wide events.
The similar qualitative shape across regular and scale-free modules supports the interpretation of an ``optimal coupling'' rather than a cutoff-specific artifact.
For a detailed branching-process description and analytic estimates of the optimal coupling $p^*$, we refer to~\cite{brummitt2012suppressing}.

\clearpage

