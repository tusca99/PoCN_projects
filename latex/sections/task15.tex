\chapter{Task 15: Self-Organized Criticality on Networks}


\section{Framework}
We simulate the Bak--Tang--Wiesenfeld (BTW) sandpile on a network with $N$ nodes.
Each node $i$ carries an integer load $z_i\in\{0,1,2,\dots\}$ and a threshold $z_c(i)$.
At each driving step one grain is added to a uniformly random node; the system then relaxes via topplings.
On a generic graph we set $z_c(i)=k_i$ (degree threshold), so a toppling at $i$ sends exactly one grain to each of its $k_i$ neighbors:
\begin{equation}
z_i \leftarrow z_i-k_i,\qquad z_j \leftarrow z_j+1\ \ \forall j\in\partial i.
\end{equation}
On finite networks without open boundaries, some form of dissipation is needed for a stationary state.
We use \emph{per-grain} dissipation~\cite{goh2003sandpile}: each outgoing grain is independently lost with probability $f$, so that on average a fraction $f$ of the redistributed load disappears per toppling.
An \emph{avalanche} is the full relaxation triggered by one grain addition.
We measure: size $S$ (total topplings), duration $T$ (parallel-update waves), and area $A$ (distinct toppled nodes).

\section{SOC on single topologies}

\paragraph{Mean-field baseline (Bonabeau).}
In the random-neighbor (annealed) model~\cite{bonabeau1995sandpile} each toppling redistributes grains to $k$ nodes chosen uniformly at random rather than along fixed edges.
This removes quenched structural correlations and yields mean-field exponents: the avalanche-size distribution follows $P(S)\sim S^{-3/2}$, i.e.\ the CCDF scales as $P(S\ge s)\sim s^{-1/2}$.
We verify this numerically with $N=20\,000$, $k=4$, and dissipation $\varepsilon\in\{10^{-3},3\!\times\!10^{-3},10^{-2}\}$ ($3\times10^5$ driving steps each, $2\times10^4$ transient discarded).
Fig.~\ref{fig:bonabeau_ccdf} (left) confirms the expected power-law tail; smaller $\varepsilon$ extends the scaling regime before the finite-size cutoff.

\paragraph{Quenched scale-free topology (Goh et al.).}
On a quenched scale-free network generated with the static model~\cite{goh2003sandpile} (mean degree $\langle k\rangle=4$, degree threshold $z_c(i)=k_i$), the avalanche exponent depends on the degree exponent $\gamma$.
Goh et al.\ predict $\tau_A=\gamma/(\gamma-1)$ for $2<\gamma<3$ (non-mean-field) and $\tau_A=3/2$ for $\gamma\ge3$ (mean-field), with $\tau_A$ defined via $P(A)\sim A^{-\tau_A}$.

\paragraph{Topology comparison.}
We run the BTW sandpile ($z_c(i)=k_i$, per-grain dissipation $f=10^{-4}$, $5\times10^5$ steps, $5\times10^4$ transient) on six synthetic topologies with $N\approx80\,000$: a 2D square lattice ($282\!\times\!282$, open boundary), an Erd\H{o}s--R\'{e}nyi (ER) random graph ($\langle k\rangle\!\approx\!4$), a Watts--Strogatz (WS) small-world network ($k\!=\!4$, rewiring $\beta\!=\!0.1$), a Barab\'{a}si--Albert (BA) preferential-attachment network ($m\!=\!2$), and two scale-free (SF) networks from the static model with $\gamma\!=\!2.5$ and $\gamma\!=\!3.0$.
Fig.~\ref{fig:bonabeau_ccdf} (right) overlays the resulting avalanche-size CCDFs.
The homogeneous topologies (lattice, ER, WS) cluster near the mean-field $S^{-1/2}$ guide, confirming that low degree variance leaves scaling unaffected.
In contrast, the heterogeneous networks (BA, SF) should show markedly steeper tails: degree-heterogeneity concentrates load on hubs, producing more frequent moderate cascades but suppressing the very large ones.
Goh et al.\ predict $\tau_S=\gamma/(\gamma-1)=5/3$ for $\gamma\!=\!2.5$ ($\text{CCDF}\sim S^{-2/3}$) and MF recovery for $\gamma\!\ge\!3$~\cite{goh2003sandpile}.
The qualitative ordering is confirmed---heterogeneous curves are steeper---but the quantitative separation is weaker than in the original paper, where dissipation occurs through a \emph{sink node} connected to all vertices (open boundary) rather than through per-grain stochastic loss.
With per-grain dissipation the effective loss rate of a hub scales as $\sim fk_i$, disproportionately damping high-degree nodes and partially washing out the topology-dependent exponent shift.
\enlargethispage*{1.1\baselineskip}

\begin{figure}[t]
    \centering
    \includegraphics[width=0.47\linewidth]{figures/task15/compare_ccdf_size.pdf}\hfill
    \includegraphics[width=0.47\linewidth]{figures/task15/topology_comparison_ccdf.pdf}
    \caption{Left: annealed random-neighbor model ($N=20\,000$, $k=4$) avalanche-size CCDF for three dissipation rates; dashed line marks the mean-field slope $-1/2$.
    Right: avalanche-size CCDF on six quenched topologies ($N\!\approx\!80\,000$, degree threshold, $f\!=\!10^{-4}$); homogeneous networks cluster near the MF guide while heterogeneous ones (BA, SF) show progressively steeper tails.}
    \label{fig:bonabeau_ccdf}
\end{figure}

\section{SOC with interdependence}

\paragraph{Coupled sandpiles and large cascades.}
We now consider two modules $A$ and $B$, each with $N=2000$ nodes, connected by a sparse set of inter-module edges (bridges).
Each node in module $A$ is independently selected as a bridge endpoint with probability $p$; the same number of nodes is drawn from $B$ and paired uniformly at random~\cite{brummitt2012suppressing}.
Because the threshold is degree-based, adding a bridge to node $i$ raises its threshold $z_c(i)$ by one (higher local stability), but also creates a new pathway for load to propagate between modules.

Following~\cite{brummitt2012suppressing}, we classify events by whether a \emph{large} cascade occurs in module $A$: $S_A>C$ with $C=N/2=1000$.
We also track \emph{global} cascades ($S>C_g$, $C_g=N=2000$).
Dissipation is per-toppling with $f=0.01$; each simulation runs $5\times10^5$ steps with $5\times10^4$ transient.
The ``regular'' modules are random $3$-regular graphs; the scale-free modules use the static model with tunable $\gamma$ (taking $\gamma=\infty$ recovers a homogeneous degree distribution near the ER limit).
In the SF panels, shaded bands show 95\% Wilson-score confidence intervals from aggregated event counts across 2 replicates.

Fig.~\ref{fig:brummitt_local_compare} separates cascades that remain within $A$ (local) from those triggered by activity arriving from module $B$ (inflicted), and their union (any large in $A$).
At small $p$, the few bridges divert load away from module $A$, suppressing its largest within-module cascades.
At large $p$, bridges raise node thresholds (making individual topplings less frequent) but simultaneously enable inflicted cascades: when a node does topple, it can inject load into the other module.
The competition between these two effects can produce a non-trivial optimum $p^*$ at which large cascades are least likely~\cite{brummitt2012suppressing}.

\begin{figure}[t]
    \centering
    \includegraphics[width=0.47\linewidth]{figures/task15/pr_large_in_A_vs_p_pretty_R3.pdf}\hfill
    \includegraphics[width=0.47\linewidth]{figures/task15/pr_large_in_A_vs_p_pretty_SF.pdf}
    \caption{Probability of a large cascade in module $A$ versus coupling $p$. Left: regular modules ($z=3$) showing the local/inflicted/overall decomposition. Right: SF modules for several $\gamma$; lower $\gamma$ shifts the minimum to smaller $p$.}
    \label{fig:brummitt_local_compare}
\end{figure}

\begin{figure}[t]
    \centering
    \includegraphics[width=0.47\linewidth]{figures/task15/pr_large_global_vs_p_pretty_R3.pdf}\hfill
    \includegraphics[width=0.47\linewidth]{figures/task15/pr_large_global_vs_p_pretty_SF.pdf}
    \caption{Probability of a \emph{global} large cascade ($S>C_g=2000$) versus $p$. The non-monotonic shape persists, supporting the ``optimal coupling'' interpretation rather than a cutoff artifact.}
    \label{fig:brummitt_global_compare}
\end{figure}
\enlargethispage*{1.2\baselineskip}
Global cascades (Fig.~\ref{fig:brummitt_global_compare}) show the same qualitative non-monotonic pattern.
Small coupling suppresses the largest events; strong coupling enables system-wide propagation.
For the regular-module case, the optimum lies near $p^*\approx 0.01$, consistent with~\cite{brummitt2012suppressing}.
In the SF extension, the optimum $p^*$ shifts to smaller values for lower $\gamma$: heterogeneous networks are more sensitive to inter-module coupling, likely because high-degree hubs act as efficient conduits for cross-module load transfer once bridges reach them.

\clearpage

