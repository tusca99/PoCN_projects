\chapter{Task 44: Social Connectedness Index II}

\section{Dataset and network construction}
The Social Connectedness Index (SCI) by Meta quantifies the intensity of Facebook friendship ties between pairs of administrative regions~\cite{metaSCI}.
For regions $i,j$ the published score is
\[
\mathrm{SCI}_{ij}\;\propto\;\frac{F_{ij}}{U_i\,U_j},
\]
where $F_{ij}$ counts cross-region friendships and $U_i,U_j$ are the Facebook-user populations; values are rescaled within each layer to $[1,\,10^{9}]$ (\texttt{scaled\_sci})~\cite{hdxSCI}.
We use the SCI~II release (reference period 2025-12-26 to 2026-01-25, CC0)~\cite{hdxSCI} and, following task instructions, exclude the USA.

\paragraph{Resolution.}
We adopt the finest resolution available in the SCI layer files: \textbf{NUTS\,3} for EU countries (via GISCO 2024 boundaries~\cite{giscoNUTS2024}) and \textbf{GADM level\,1} for all others (GADM~v4.1~\cite{gadm41}).
The SCI data do not provide sub-national codes at GADM level~3; GADM\,1 is therefore the highest resolution accessible outside Europe.

\paragraph{Construction.}
From the raw layer CSVs we retain only \emph{within-country} edges, drop self-loops, keep one edge per unordered pair, and sum duplicate \texttt{scaled\_sci} values.
Geographic coordinates (EPSG:4326) are obtained from boundary-polygon representative points.
The resulting global deliverable contains $N=3\,040$ nodes and $E=134\,016$ edges across 99 countries; coordinate coverage is ${\sim}\,98\%$ of nodes.

\section{Completeness and weighted perspective}

A key observation is that \emph{every} within-country graph is \textbf{complete}: the SCI dataset reports a weight for every pair of regions, so $E=\binom{N}{2}$ and density$\,=1$ for all 99 countries (Fig.~\ref{fig:task44_sanity}, left).
Consequently, standard unweighted metrics are uninformative: $P(k)=\delta_{k,N-1}$, $C=1$, and modularity $Q=0$.
All meaningful structure resides in the \emph{weights}; the remainder of the analysis is therefore weighted.

\begin{figure}[t]
    \centering
    \includegraphics[width=0.49\linewidth]{figures/task44/scatter_N_vs_E.pdf}\hfill
    \includegraphics[width=0.49\linewidth]{figures/task44/mean_scaled_sci_rank_topK.pdf}
    \caption{Left: edges versus nodes per country (log--log); the black curve is the complete-graph reference $E=N(N-1)/2$, confirming density$\,=1$ for all countries. Right: mean within-country \texttt{scaled\_sci} per edge versus country rank (top~10).}
    \label{fig:task44_sanity}
\end{figure}

\paragraph{Edge-weight distribution.}
Within each country we normalise weights by the maximum ($\hat w_{ij}=w_{ij}/w_{\max}$) and pool all edges across the 99 countries.
The resulting CCDF (Fig.~\ref{fig:weight_strength}, left) spans roughly four decades with a heavy tail: most edges carry low relative weight while a few region pairs are orders of magnitude more connected.

\paragraph{Node strength and Gini coefficient.}
The strength of node $i$ is $s_i=\sum_j w_{ij}$.
We summarise within-country strength heterogeneity via the Gini coefficient $G\in[0,1]$ (Fig.~\ref{fig:weight_strength}, right).
Values range from $G\approx0.07$ (Japan, very homogeneous) to $G\approx0.35$ (UK, dominated by London);
the colour encodes mean strength, showing that heterogeneity is not simply a size effect.

\begin{figure}[t]
    \centering
    \includegraphics[width=\linewidth]{figures/task44/weight_and_strength.pdf}
    \caption{Left: CCDF of the max-normalised edge weight pooled across all 99 countries. Right: strength Gini coefficient versus network size; colour indicates $\log_{10}\langle s\rangle$.}
    \label{fig:weight_strength}
\end{figure}

\section{Weighted clustering}

For complete graphs the unweighted clustering coefficient is trivially~1.
We therefore compute the \emph{weighted} clustering of Onnela et al.~\cite{onnela2005intensity}:
\[
C_i^w \;=\; \frac{1}{s_i\,(k_i-1)}\sum_{j,h}\bigl(\hat w_{ij}\,\hat w_{ih}\,\hat w_{jh}\bigr)^{1/3},
\]
where $\hat w$ is the max-normalised weight.
$C^w$ equals~1 only when all triangle weights are identical (perfectly homogeneous connectivity); heterogeneous weights lower it.

Fig.~\ref{fig:clustering} (left) shows $\langle C^w\rangle$ per country (computed for countries with $N\le120$ for tractability).
Russia, Japan, and Uganda ($C^w\!\gtrsim\!0.89$) exhibit nearly uniform regional connectivity, whereas the UK ($C^w\!\approx\!0.39$) is strongly London-centric.
The weight matrix of Italy (Fig.~\ref{fig:clustering}, right) visualises the block structure typical of countries with a few dominant metropolitan areas.

\begin{figure}[t]
    \centering
    \includegraphics[width=\linewidth]{figures/task44/clustering_comparison.pdf}
    \caption{Left: weighted clustering $\langle C^w\rangle$ versus $N$; dashed line marks $C^{\rm unw}=1$. Right: log-scaled weight matrix for Italy ($N=107$, NUTS\,3 regions).}
    \label{fig:clustering}
\end{figure}

\clearpage