\chapter{Task 44: Social Connectedness Index II}

\section{Dataset and goal}
We use the Social Connectedness Index II (SCI) by Meta/HDX, which reports the intensity of Facebook friendship ties between pairs of administrative regions.
Each record provides a pair of regions and a positive score (\texttt{scaled\_sci}).
This release covers 178 countries and corresponds to the reference period 2025-12-26 to 2026-01-25 (updated 2026-02-07, CC0) \cite{hdxSCI}.
Following the task instructions, we exclude the USA from our outputs.

\paragraph{Concept of SCI.}
SCI is a normalized measure of friendship intensity between locations. Conceptually, for regions $i,j$ one can think of
\[
\mathrm{SCI}_{ij} \propto \frac{F_{ij}}{U_i\,U_j},
\]
where $F_{ij}$ is the number of friendships connecting the two regions and $U_i,U_j$ are the corresponding Facebook-user populations; published values are then rescaled within each layer to $[1,10^9]$ (\texttt{scaled\_sci}) \cite{hdxSCI,metaSCI}.
In our analysis we treat \texttt{scaled\_sci} as a \emph{within-country} edge weight.

\section{Network construction}
\paragraph{End-to-end pipeline.}
From the raw SCI layer CSVs, we:
\begin{itemize}
    \item select the set of countries to include;
    \item set the resolution per country: EU countries use NUTS3 (higher resolution), while all other countries use GADM level 1;
    \item keep within-country rows only (\texttt{user\_country = friend\_country}), drop self-loops, store one edge per unordered pair, and sum duplicate \texttt{scaled\_sci};
    \item optionally add coordinates by joining a separate region-centroids table (built from boundary polygons), then export the submission CSVs and run sanity checks.
\end{itemize}

\paragraph{Nodes.}
Nodes represent subnational administrative regions at the best available resolution.
We export a single \texttt{nodes.csv} with global consecutive \texttt{nodeID} and \texttt{nodeLabel}; labels are prefixed with ISO3 (e.g., \texttt{ITA:...}) to avoid collisions.

\paragraph{Edges.}
Edges are built \emph{within country} only.
We treat ties as undirected, keep one edge per unordered region pair, and aggregate duplicates by summing \texttt{scaled\_sci}; self-loops are removed.
The submission file \texttt{edges.csv} stores endpoints and country tags; \texttt{edges\_weighted.csv} is produced for analysis/debug.

\paragraph{Geographic coordinates.}
The SCI layer CSVs do not include coordinates.
We therefore build a region-to-coordinate table by downloading administrative boundary datasets and extracting one representative point per polygon (EPSG:4326), using:
\begin{itemize}
    \item \textbf{GADM v4.1} level-1 polygons (downloaded as shapefiles; region code \texttt{GID\_1}) \cite{gadm41}.
    \item \textbf{EU NUTS 2024} level-3 polygons (GISCO GeoJSON; region code \texttt{NUTS\_ID}) \cite{giscoNUTS2024}.
\end{itemize}
The resulting \texttt{(latitude, longitude)} are merged into \texttt{nodes.csv} when available; boundary files are large and are downloaded on demand and cached locally.

\section{Sanity checks}
For the selected top-100 countries (by available GADM1 regions, USA excluded), the global output contains $N=3040$ nodes and $E=134016$ edges.
Coordinate coverage is high after including centroids (\mbox{$\sim 98\%$ of nodes}).
Within-country graphs are often very dense (close to complete), so unweighted structure is less informative than weighted summaries.
To mitigate size effects we inspect mean edge weight (total \texttt{scaled\_sci} divided by $E$).

\begin{figure}[h]
    \centering
    \includegraphics[width=0.47\linewidth]{figures/task44/scatter_N_vs_E.pdf}\hfill
    \includegraphics[width=0.47\linewidth]{figures/task44/mean_scaled_sci_rank_topK.pdf}
    \caption{Left: number of edges versus number of nodes for each country network (log--log scale). The black curve is the complete-graph reference $E=N(N-1)/2$. Right: mean within-country connectedness (total \texttt{scaled\_sci} divided by $E$) versus country rank (top 10).}
    \label{fig:task44_sanity}
\end{figure}

\clearpage